%%%%%%%%%%%%%%%%%%%%%%%%%%%%%%%%%%%%%%%%%%%%%%%%%%%%%%%%%%%%%%
% PREENCHA AS INFORMAÇÕES NECESSÁRIAS
%%%%%%%%%%%%%%%%%%%%%%%%%%%%%%%%%%%%%%%%%%%%%%%%%%%%%%%%%%%%%%

%%% Título 
\newcommand{\TítuloDoTrabalho}{%
    Decodificando a Semântica Espacial: Uma Análise Comparativa da Performance de LLMs de Código Aberto em Comparação com Sistemas NMT na Tradução de Legendas do EN-PT-br}

%%% Título em inglês
\newcommand{\TítuloEmIngles}{%
    Decoding Spatial Semantics: A Comparative Analysis of the Performance of Open-source LLMs against NMT Systems in Translating EN-PT-br Subtitles}

%%% Versão original ou corrigida? Escolha.
\newcommand{\Versao}{Versão 
Corrigida
% Corrigida
}

%%% Autor (nome completo por extenso)
\newcommand{\Autor}{%
  Rafael Macário Fernandes 
}

%%% Qual é seu último nome? (Como você usa nas publicações). Se houver
%%% Júnior, Filho etc., indique apropriadamente. Ex: Silva Filho.
\newcommand{\Sobrenome}{%
  Fernandes}

%%% Qual é seu primeiro nome, nome do meio, e demais partes do
%%% sobrenome (não incluídas acima)? A soma deste item e do anterior
%%% devem formar seu nome completo por extenso. Não abrevie.
\newcommand{\Nome}{%
  Rafael}

%%% Escreva as iniciais do seu nome *sem o sobrenome* com a pontuação correspondente,
%%% tal como você quer ver na Ficha Catalográfica
\newcommand{\IniciaisDoNome}{%
  R.}


%%%% Indique o nome do seu Orientador (nome completo por extenso)
\newcommand{\Orientador}{%
  Marcos Lopes}

%%% Orientadora ou Orientador? Selecione umas das linhas abaixo. Não remova as barras invertidas, apenas selecione a linha apropriada, comentando a outra
\newcommand{\DoutorOuDoutora}{%
  Prof.\ Dr.\
  %Prof\textsuperscript{a} Dr\textsuperscript{a}
}

%%%% Dizeres da titulação
\newcommand{\TítuloObtido}{%
Dissertação de Mestrado apresentada à Faculdade de Filosofia, Letras e Ciências Humanas da Universidade de São Paulo como um dos requisitos para a obtenção do título de Mestre em Letras.}


%%%% Ano de depósito do trabalho
\newcommand{\AnoDepósito}{%
  2024}

%%%% Número total de páginas do trabalho
\newcommand{\NumPáginas}{%
  90}

%%%% Tipo de trabalho
\newcommand{\TipoTrabalho}{%
  Dissertação (Mestrado em Linguística)}


%%%% Indique as palavras-chave para a FICHA CATALOGRÁFICA
% Cada uma delas deve ser precedida por um número com ponto.
% Todas palavras devem iniciar com letras maiúsculas
\newcommand{\PalavrasChaveFicha}{%
  1.~Natural Language Processing (NLP)~~2.~Open-source Large Language Models (LLMs)~~3.~Neural Machine Translation (NMT)~~4.~Spatial Semantics~~5.~Preposition Polysemy~~6.~Language Typology}



%%% Palavras chave para Resumo e Abstract

% Resumo
% Use a mesma lista indicada acima
% Todas palavras devem iniciar com letras maiúsculas e terminar com um ponto final.
\newcommand{\PalavrasChave}{%
  Natural Language Processing (NLP). Open-source Large Language Models (LLMs). Neural Machine Translation (NMT). Machine Translation Evaluation. Spatial Semantics. Preposition Polysemy. Language Typology}



%%%% Palavras-chave em inglês.
% Siga as mesmas regras das palavras-chave em português.
\newcommand{\Keywords}{%
  Processamento de Linguagem Natural (NLP). Modelos de Linguagem (LLMs). Tradução Automática Neural (NMT). Avaliação da Tradução Automática. Semântica Espacial. Polissemia das Preposições. Tipologia Linguística.}