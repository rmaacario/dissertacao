\begin{brazil}
\begin{abstract}
\sloppy
\noindent \MakeUppercase{\Sobrenome}, \IniciaisDoNome ~\textit{\TítuloDoTrabalho.} \TipoTrabalho. Faculdade de Filosofia, Letras e Ciências Humanas, Universidade de São Paulo, \AnoDepósito.\\

\bigskip


% Insira aqui o seu resumo em português (em um único parágrafo de 150 a 500 palavras). Deve apresentar os objetivos, métodos, resultados e conclusões.
\noindent % Não tire esse comando dessa posição, colado ao texto.
Esta dissertação de mestrado investiga os desafios da tradução da espacialidade usando Grandes Modelos de Linguagem (LLMs) de código aberto em comparação com sistemas tradicionais de Tradução Automática Neural (NMT), abordando problemas na tradução de preposições espaciais como ACROSS, INTO, ONTO e THROUGH, que frequentemente são traduzidas utilizando-se formas verbais ou preposicionais semelhantes no português (EN-PT-br). A tradução correta dessas preposições é crucial para manter a integridade semântica da língua de origem, garantindo fluidez e aderência aos padrões de lexicalização da língua-alvo \parencite{house2018, talmy2000towardb, slobin2005relating}. A pesquisa contextualiza os desafios da tradução da linguagem espacial, destacando as limitações dos sistemas NMT atuais e as potenciais vantagens dos LLMs. A revisão de literatura traça a evolução das teorias de tradução, o desenvolvimento da NMT e o surgimento dos LLMs, descrevendo também suas limitações. A metodologia emprega uma análise baseada em corpus, a partir de um conjunto de dados bilíngue centrado em preposições espaciais de legendas de TED Talks obtidos pela plataforma OPUS. Este conjunto de dados foi meticulosamente pré-processado para facilitar tanto o cálculo de métricas automatizadas quanto a análise de erros manual. As métricas utilizadas incluem BLEU, METEOR, BERTScore, COMET e TER, enquanto a análise manual identifica e categoriza os tipos de erros que cada sistema comete. Os resultados revelam que LLMs de tamanho moderado, como LLaMa-3-8B e Mixtral-8x7B, alcançam precisão próxima aos sistemas NMT, como o Google, embora essa relação nem sempre seja linear, pois modelos como Gemma-7B possuíram desempenho similar na avaliação humana. No entanto, os LLMs em geral apresentaram sérios erros de tradução, incluindo interlíngua/code-switching (in) e anglicismos (an), não conseguindo transmitir idiomaticidade na língua-alvo. Por outro lado, os sistemas NMT alcançaram muito melhor fluidez na tarefa de tradução automática. No entanto, a análise humana destaca os desafios contínuos enfrentados tanto pelos LLMs quanto pelos sistemas NMT na tradução das nuances da espacialidade, com ambos os grupos apresentando números consistentes de erros como polissemia (po) e projeção sintática (sp), nos quais falham em traduzir o significado apropriado de uma preposição ou copiam os padrões de lexicalização da língua de origem para o texto alvo \parencite{fernandes-etal-2024-spatial, oliveira2022expressing}. A dissertação conclui que, apesar dos avanços nos LLMs, permanecem obstáculos na tradução precisa da linguagem espacial, sugerindo que pesquisas futuras devem se concentrar em aprimorar conjuntos de dados de treinamento, refinar arquiteturas desses modelos e desenvolver métricas de avaliação mais sofisticadas que capturem melhor as sutilezas da semântica espacial. Este estudo contribui para o campo fornecendo uma comparação detalhada do desempenho de LLMs e NMT na tradução da linguagem espacial do EN-PT-br, propondo direções para melhorias futuras.

\bigskip \noindent
\textbf{Palavras-chave:} \PalavrasChave .
\end{abstract}
\end{brazil}